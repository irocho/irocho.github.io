Se por calquera razón un programa precisa o uso do hardware directamente pódese inserir unhas liñas de código para acceder a funcións concretas que se denominan  \textit{chamadas ó sistema}. Son os fabricantes os que proporcionan esa información en bibliotecas chamadas API. Se quero programar un videoxogo pode ser que teña que acceder á API de Windows e empregar as funcións que me proporciona Microsoft. En principio a relación co hardware só é responsabilidade do sistema operativo.

Para que o programador non se teña que ocupar dos detalles do hardware e non se poda estragar nada,  a maioría das computadoras teñen dous modos de operación: modo kernel e modo usuario.


\begin{itemize}
\item 
O sistema operativo é a peza fundamental do software e  execútase en modo kernel (ou modo supervisor). Neste modo tense acceso completo a todo o hardware e pódese executar calquera instrución que a máquina sexa capaz.
\item O resto do software  execútase en modo usuario: só un subconxunto de instrucións están permitidas. Están prohibidas para os programas que se executan neste modo as que afectan especialmento ó control da máquina ou as que se encargan da entrada e saída de información 
\end{itemize}


\begin{diapo}\begin{frame}{O sistema operativo execútase en modo \dots}
\begin{itemize}
\item hardware\pause
\item usuario \pause
\item supervisor 
\end{itemize}
\end{frame}\end{diapo} 
%parella
\begin{diapo}\begin{frame}{As instrucións que pode executar un sistema operativo son \dots}
\begin{enumerate}
\item só as funcións do modo aplicación \pause
\item calquera das instrucións da máquina \pause
\item un subconxunto das instrucións da máquina
\end{enumerate} 
\end{frame} 
\end{diapo} 
\note{despois das distros}


\begin{diapo}\begin{frame}{O modo usuario pode executar \dots}
\begin{enumerate}
\item só as funcións do modo aplicación \pause
\item calquera das instrucións da máquina \pause
\item un subconxunto das instrucións da máquina
\end{enumerate} 
\end{frame} 
\end{diapo} 
%parella
\begin{diapo}\begin{frame}{O modo usuario ten prohibido  executar \dots}
\begin{enumerate}
\item  as instrucións de control da máquina \pause
\item  as instrucións de entrada/ saída\pause
\item todas as respostas son correctas
\end{enumerate} 
\end{frame} 
\end{diapo} 

