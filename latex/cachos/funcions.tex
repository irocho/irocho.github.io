Hoxe en día un sistema operativo é un software moi complexo que permite que o hardware sexa \textit{transparente} para un programador, é dicir, non hai que se preocupar da marca do escáner  se queremos obter unha imaxe na pantalla. Salvo cando se programa a moi baixo nivel (con instrucións en código máquina) o resto dos programas poden executarse para o hardware de calquera máquina na que estea instalado o mesmo  sistema operativo. O sistema operativo é xa que logo o organizador e o xestor de todo o que se fai: o obxectivo é facilitar o  uso do ordenador. 

\begin{diapo} \begin{frame}{ Un programa feito para Windows    \dots} 
\begin{enumerate}
	\item non se pode executar nun Linux\pause
	\item non se pode executar no MacOS \pause
	\item todas as respostas son correctas 
\end{enumerate} \end{frame}  \end{diapo}  
%parella
\begin{diapo}\begin{frame}{ Se algo é \textit{transparente} ó usuario significa  \dots}
\begin{enumerate}
	\item  que o usuario debe estar informado do que fai ese hardware\pause
	\item que o usuario ten que se executar un programa  especial para que se vexa \pause
	\item que o usuario non se ten que preocupar 
\end{enumerate} \end{frame} \end{diapo}
\begin{singlespace}
Os sistemas operativos actuais teñen encomendadas moitas  misións, moitas son familiares para nós e seguro que todos as usamos a cotío:

\begin{description}
\item[Simplificar a relación co usuario:] Sexa cunha interfaz modo texto ou modo gráfico.
\item[Controlar a execución dos programas: ] Aceptar os traballos, administrar o xeito no que se realizan, asignar recursos e finalízalos cando cómpra.
\item[Xestionar os sistemas de arquivo:] manter a lista de arquivos do disco e favorecer a súa organización (por exemplo en directorios) e a súa manipulación (creación, modificación, eliminación, etc)
\item[Administrar periféricos:] Coordinar e organizar os dispositivos conectados ó ordenador. Con que eu faga \texttt{Arquivo/Imprimir} podo pasar a papel os meus documentos sen preocuparme do funcionamento dos rodillos da impresora.
\end{description}


\begin{diapo}\begin{frame}{Para interacionar cun sistema operativo podo usar  \dots}
\begin{enumerate}
\item  modo texto \pause
\item GUI \pause
\item entorno gráfico  \pause
\item todas as respostas son correctas
\end{enumerate}
\end{frame}
\end{diapo}
\begin{diapo}
\begin{frame}{As chamadas ó sistema \dots}
\begin{enumerate}
\item  son ficheiros de audio \pause
\item son un entorno gráfico \pause
\item son funcións tipo API  \pause
\item todas as respostas son correctas
\end{enumerate}
\end{frame}
\end{diapo}

Outras funcións teñen un carácter máis técnico e serán as que traballaremos máis polo miúdo:
\begin{description} 
\item[Xestión de permisos e usuarios:] Adxudica permisos de acceso e evita que as accións dun usuario afecten ó traballo que fai outro. Ou que un usuario cotillee nos documentos de outro sen permiso.
\item[Control de concurrencia:] Establece prioridades cando se precise usar un recurso. Se varios programas teñen que usar o procesador non pode ser que o usen todos á vez e que se mesturen os datos.
\item[Administración de memoria:] Asigna posicións de memoria e xestiona o seu uso. Non necesito preocuparme das posicións de memoria que estou ocupando.
\item[Control de seguridade:] Garantiza que a información se almacene dun xeito seguro. Uns datos non poden pisar ós outros.
\item[Apoio a programas:] permitindo o uso de servizo dispoñibles ou chamadas ó sistema.
\item[Control de erros:] Xestiona os erros de hardware e a perda de datos. 
\end{description}
\end{singlespace}


\begin{diapo}\begin{frame}{O sistema operativo ten coma función \dots}
\begin{enumerate}
\item xestionar os recursos da computadora \pause
\item executar servizos para os programas\pause
\item todas as respostas son correctas
\end{enumerate} 
\end{frame} 
\end{diapo}
%parella
\begin{diapo}\begin{frame}{O sistema operativo ten coma función \dots}
\begin{enumerate}
\item executar mandatos de usuarios \pause
\item executar servizos para os programas\pause
\item todas as respostas son correctas
\end{enumerate} 
\end{frame}
\end{diapo}

\begin{diapo}\begin{frame}{O sistema operativo ten coma función \dots}
\begin{enumerate}
\item xestionar os recursos da computadora \pause
\item executar mandatos de usuarios \pause
\item executar servizos para os programas\pause
\item todas as respostas son correctas
\end{enumerate} 
\end{frame}
\end{diapo}

