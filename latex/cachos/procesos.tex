Un proceso é un programa en execución. Cada proceso componse do código que se executa e a correspondente estructura de datos. Ambos estarán cargados en memoria e terán uns recursos asignados: espacio en memoria, uso da CPU, etc.  O sistema operativo é o encargado de controlar a execución.\\

\begin{diapo}\begin{frame}{Un proceso é \dots}
\begin{enumerate}
\item  un programa almacenado en memoria \pause
\item  un ficheiro de datos\pause
\item un programa que se está executando
\item ningunha das respostas é correcta
\end{enumerate} 
\end{frame} 
\end{diapo} 
%parella
\begin{diapo}\begin{frame}{Un proceso está formado por \dots}
\begin{enumerate}
\item  o código que se executou\pause
\item  os datos que precisa para a súa execución \pause
\item as anteriores respostas son correctas
\item ningunha das respostas é correcta
\end{enumerate} 
\end{frame} 
\end{diapo} 


\begin{diapo}\begin{frame}{Quen controla a execución dun proceso é \dots}
\begin{enumerate}
\item  a memoria \pause
\item  a CPU\pause
\item os recursos de hardware cando se solicitan
\item ningunha das respostas é correcta
\end{enumerate} 
\end{frame} 
\end{diapo} 
%parella
\begin{diapo}\begin{frame}{Quen asigna os recursos que precisa un proceso é \dots}
\begin{enumerate}
\item  a memoria \pause
\item  o sistema operativo \pause
\item os recursos de hardware cando se solicitan
\item ningunha das respostas é correcta
\end{enumerate} 
\end{frame} 
\end{diapo} 


\begin{diapo}\begin{frame}{Cando se executa un proceso \dots}
\begin{enumerate}
\item  permanece en memoria \pause
\item  permanece no sistema operativo \pause
\item os recursos de hardware quedan bloqueados
\item ningunha das respostas é correcta
\end{enumerate} 
\end{frame} 
\end{diapo} 
%parella
\begin{diapo}\begin{frame}{O responsable da finalización dun proceso é \dots}
\begin{enumerate}
\item  a memoria \pause
\item  a CPU\pause
\item o sistema operativo
\item ningunha das respostas é correcta
\end{enumerate} 
\end{frame} 
\end{diapo} 


\begin{singlespace}
O contido da estructura de datos dun proceso que  permite controlar todos os aspectos da súa execución é:
\begin{description}
	\item[Estado actual do proceso:] Pode estar en execución, agardando, parado,..

	\item[Identificación:] Os procesos teñen cadanseu PID ou sexa un número que permite que o sistema operativo poda identificalo. 
	\item[Prioridade:] Número que indica a vez para a súa execución. O que teña maior prioridade dos que están agardando executarase antes.
	\item[Zona de memoria:] Cada proceso ten reservado un espacio en memoria que non pode ser ocupado por outros procesos.
	\item[Recursos asociados:] Un proceso ten necesidades propias que ten que coñecer o sistema operativo, por exemplo o acceso a un ficheiro  determinado.
\end{description} 
\end{singlespace}

\begin{diapo} \begin{frame}{ A estructura de datos dun proceso   \dots} 
\begin{enumerate}
	\item controla os datos que necesita o programa\pause
	\item contén información do proceso \pause
	\item é unha lista do hardware disponible e conectado 
\end{enumerate} \end{frame}  \end{diapo}  
%parella
\begin{diapo}\begin{frame}{ A estructura de datos dun proceso está  \dots}
\begin{enumerate}
	\item  almacenada no disco duro \pause
	\item  en memoria dende que se encende o ordenador \pause
	\item en memoria mentres estea o proceso en memoria 
\end{enumerate} \end{frame} \end{diapo}



\begin{diapo} \begin{frame}{ A prioridade dun proceso é   \dots} 
\begin{enumerate}
	\item unha palabra \pause
	\item unha función \pause
	\item un número 
\end{enumerate} \end{frame}  \end{diapo}  
%parella
\begin{diapo}\begin{frame}{ Un proceso pode   \dots}
\begin{enumerate}
	\item instalarse nunha posición de memoria ocupada\pause
	\item solapar outro proceso se está agardando un recurso \pause
	\item  ocuupar posicións de memoria libres
\end{enumerate} \end{frame} \end{diapo}

\begin{diapo} \begin{frame}{ O PID ten que ver con   \dots} 
\begin{enumerate}
	\item identificación do hardware\pause
	\item  identificación do software \pause
	\item distinguir procesos 
\end{enumerate} \end{frame}  \end{diapo}  
%parella
\begin{diapo}\begin{frame}{ hardware   \dots}
\begin{enumerate}
	\item drivers \pause
	\item terminais \pause
	\item sistemas 
\end{enumerate} \end{frame} \end{diapo}
Un proceso pódese crear, executar, poñelo en espera ou matalo. Existen uns procesos que se crean no arranque do sistema eqeu permanecen en segundo plano e son os que están pendentes do correo electrónico, de que se imprima correctamente ou de avisar de eventos da axenda. Estes procesos en Linux chámanse \textit{demos}. Se quixéramos crear a man un proceso neste sistema operativo temos o comando \texttt{fork} e para monitorizar os procesos que se están executando usaremos \texttt{top, ps}. Se queremos rematar un proceso empregaremos \texttt{kill} indicando o PID. Moito olliño con facelo sen estar seguros do que facemos.\\

\begin{diapo}\begin{frame}{ Se quero ver en tempo real os procesos que está executando o meu Linux debo usar \dots}
\begin{enumerate}
	\item  o comando \texttt{xeyes}
	\pause
	\item o comando \texttt{cal}
	\pause
	\item  o comando \texttt{top}
	\pause
	\item   calquera dos comandos anteriores é válido
\end{enumerate} 
\end{frame} 
\end{diapo} 
%parella
\begin{diapo}\begin{frame}{ Se quero rematar un proceso debo usar o comando\dots}
\begin{enumerate}
	\item   \texttt{xeyes}
	\pause
	\item \texttt{kill}
	\pause
	\item   \texttt{man}
	\pause
	\item   calquera dos comandos anteriores é válido
\end{enumerate} 
\end{frame} 
\end{diapo} 
Cada proceso ocupa un espazo propio en memoria. Con frecuencia é conveniente ter varios fíos de control no mesmo espazo de direccións de memoria que comparten os datos e  que se executan á vez. Desenvolvemos así varias actividades conxuntamente e algunhas  pódense bloquear sen necesidade de bloquear todo o proceso. Descompoñer unha aplicación en varios fíos  que se executan casi en paralelo mellor a eficiencia do sistema. Por exemplo se temos varios procesadores cada un podería executar un fío sen ter que agardar que se execute un tras outro.\\

\begin{diapo} \begin{frame}{ Os fíos dun proceso   \dots} 
\begin{enumerate}
	\item teñen cadanseu espazo de memoria\pause
	\item comparten espazo de memoria \pause
	\item non usan memoria, execútanse directamente 
\end{enumerate} \end{frame}  \end{diapo}  
%parella
\begin{diapo}\begin{frame}{ Usando fíos   \dots}
\begin{enumerate}
	\item melloramos a rapidez dun programa \pause
	\item obtemos un certo paralelismo \pause
	\item todas as respostas son correctas 
\end{enumerate} \end{frame} \end{diapo}

